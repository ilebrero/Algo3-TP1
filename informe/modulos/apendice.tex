\newpage
\section{Apendice}
\subsection{Codigo Ejercicio1}
\lstinputlisting[language=Java]{modulos/Tests_Code/ejercicio1.txt}
\subsection{Codigo Ejercicio2}
\lstinputlisting[language=Java]{modulos/Tests_Code/ejercicio2.txt}
\subsection{Codigo Ejercicio3}
\lstinputlisting[language=Java]{modulos/Tests_Code/ejercicio3.txt}
\subsection{Informe de modificaciones}
\subsubsection{Ejercicio 1}
Agregamos el enunciado del problema al principio del ejercicio.\\
Modificamos los gr\'aficos.
\subsubsection{Ejercicio 2}
Se modific\'o la secci\'on de correctitud de \'este problema.
\subsubsection{Ejercicio 3}
En el c\'odigo:

Dentro de la recursi\'on ya no chequeamos que la amiga m\'as lejana sea menor.\\
El objeto ronda calcula la amiga m\'as lejana y la suma de distancias al mismo tiempo con la funci\'on "$calcularDistancias$".\\
En el informe:

Se hicieron cambios respecto a la descripci\'on del algoritmo para explicar c\'omo funciona ahora el mismo.\\
Se modific\'o la tabla de valores en d\'onde most\'abamos la performance del mejor vs el peor caso.\\
Se agreg\'o el enunciado al principio del ejercicio.\\
Se modificaron los gr\'aficos.